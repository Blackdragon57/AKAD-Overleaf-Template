\chapter{Grundlagen}
\section{Text}
Das ist eine \ac{abk}, die bei der zweiten Verwendung nur noch in der Kurzform \ac{abk} angezeigt wird.

\subsection{Fett}
\textbf{Lorem ipsum dolor sit amet, consetetur sadipscing elitr, sed diam nonumy eirmod tempor invidunt ut labore et dolore magna aliquyam erat, sed diam voluptua.}

\subsection{Kursiv}
\textit{Lorem ipsum dolor sit amet, consetetur sadipscing elitr, sed diam nonumy eirmod tempor invidunt ut labore et dolore magna aliquyam erat, sed diam voluptua.}

\subsection{Unterstrichen}
orem ipsum \underline{dolor} sit amet, consetetur sadipscing elitr, sed diam nonumy eirmod \underline{tempor} invidunt ut labore et dolore magna \underline{aliquyam} erat, sed diam voluptua.

\section{Fußnote}

Text mit Fußnote \footnote{Die Fußnote zum Text} 

\section{Zitate}

Dies ist ein ganz kurzer Beispieltext \footnote{\cite{Baeumle-Courth2004}}. Und noch ein weites Zitat \footnote{\cite{Torvalds2001}}
\\
Zitate auf Webseiten \footnote{\cite{gabler:individualsoftware}} \footnote{\cite{gabler:standardsoftware}}
\\
http://www.literatur-generator.de/

\section{Aufzählung}

\begin{itemize}
\item\textit{Punkt 1:} Text
\item Punkt 2: Text
\item Punkt 3: \\ Text
\end{itemize}

\section{Abkürzungen}
Hier werden \ac{Abk} aus dem Verzeichnis aufgerufen und gleich nochmal die gleiche \ac{Abk}.